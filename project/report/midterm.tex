%% bare_conf_compsoc.tex
%% V1.4b
%% 2015/08/26
%% by Michael Shell
%% See:
%% http://www.michaelshell.org/
%% for current contact information.
%%
%% This is a skeleton file demonstrating the use of IEEEtran.cls
%% (requires IEEEtran.cls version 1.8b or later) with an IEEE Computer
%% Society conference paper.
%%
%% Support sites:
%% http://www.michaelshell.org/tex/ieeetran/
%% http://www.ctan.org/pkg/ieeetran
%% and
%% http://www.ieee.org/

%%*************************************************************************
%% Legal Notice:
%% This code is offered as-is without any warranty either expressed or
%% implied; without even the implied warranty of MERCHANTABILITY or
%% FITNESS FOR A PARTICULAR PURPOSE! 
%% User assumes all risk.
%% In no event shall the IEEE or any contributor to this code be liable for
%% any damages or losses, including, but not limited to, incidental,
%% consequential, or any other damages, resulting from the use or misuse
%% of any information contained here.
%%
%% All comments are the opinions of their respective authors and are not
%% necessarily endorsed by the IEEE.
%%
%% This work is distributed under the LaTeX Project Public License (LPPL)
%% ( http://www.latex-project.org/ ) version 1.3, and may be freely used,
%% distributed and modified. A copy of the LPPL, version 1.3, is included
%% in the base LaTeX documentation of all distributions of LaTeX released
%% 2003/12/01 or later.
%% Retain all contribution notices and credits.
%% ** Modified files should be clearly indicated as such, including  **
%% ** renaming them and changing author support contact information. **
%%*************************************************************************


% *** Authors should verify (and, if needed, correct) their LaTeX system  ***
% *** with the testflow diagnostic prior to trusting their LaTeX platform ***
% *** with production work. The IEEE's font choices and paper sizes can   ***
% *** trigger bugs that do not appear when using other class files.       ***                          ***
% The testflow support page is at:
% http://www.michaelshell.org/tex/testflow/



\documentclass[conference]{IEEEtran}
%\documentclass{IEEEtran}
% Some/most Computer Society conferences require the compsoc mode option,
% but others may want the standard conference format.
%
% If IEEEtran.cls has not been installed into the LaTeX system files,
% manually specify the path to it like:
% \documentclass[conference,compsoc]{../sty/IEEEtran}


%%%%%%%%%%%%%%%%%%%%%%%%%%%%%%%%%%%%%%%%%%%%%%%%%%%%%%%%%%
% Inline comments. Pick initials and color of your choice. \ysnote{} refers to Yogesh's note. 
%
\usepackage[usenames,dvipsnames,svgnames,x11names]{xcolor}
\newcommand{\ysnote}[1]{ {\textcolor{magenta} { ***Yogesh: #1 }}} % needs a response
\newcommand{\drnote}[1]{ {\textcolor{orange} { ***Dreamer: #1 }}}
\newcommand{\Note}[1]{\textcolor{red}{#1}} % verify if this is correct
\newcommand{\ysnoted}[1]{ {\textcolor{green} { ***TODO Later: #1 }}} % postpone addressing of comment

%%%%%%%%%%%%%%%%%%%%%%%%%%%%%%%%%%%%%%%%%%%%%%%%%%%%%%%%%%
% Additional fonts
\usepackage[T1]{fontenc} %% https://tex.stackexchange.com/questions/664/why-should-i-use-usepackaget1fontenc
\usepackage{lmodern} %% https://tex.stackexchange.com/questions/2369/why-do-the-less-than-symbol-and-the-greater-than-symbol-appear-wrong-as
\usepackage[normalem]{ulem} % required for strikeout font
% Default Computer Modern font (no bold implemented)
%\renewcommand{\ttdefault}{cmtt}
% Hence, Using Courier font
\renewcommand{\ttdefault}{pcr}


%%%%%%%%%%%%%%%%%%%%%%%%%%%%%%%%%%%%%%%%%%%%%%%%%%%%%%%%%%
% Change tracking for article revisions. Added, Deleted, Replaced, or Modified content.
%
\newcommand{\modc}[1]{{\textcolor{blue}{#1}}}
\newcommand{\addc}[1]{{\textcolor{teal}{#1}}}
\newcommand{\delc}[1]{ {\textcolor{gray} {\sout{#1}} }}
%\newcommand{\delc}[1]{} % uncomment this (and comment above line) to ignore showing deletion
\newcommand{\repc}[2]{ {\textcolor{gray} {\sout{#1}} }{\textcolor{teal} {#2}}}
%\newcommand{\repc}[2]{{\textcolor{teal}{#2}}} % uncomment this (and comment above line) to ignore showing deletion
%
%---------------------------------------------------------


%%%%%%%%%%%%%%%%%%%%%%%%%%%%%%%%%%%%%%%%%%%%%%%%%%%%%%%%%%
% *** MISC UTILITY PACKAGES ***
%
%\usepackage{ifpdf}
% Heiko Oberdiek's ifpdf.sty is very useful if you need conditional
% compilation based on whether the output is pdf or dvi.
% usage:
% \ifpdf
%   % pdf code
% \else
%   % dvi code
% \fi
% The latest version of ifpdf.sty can be obtained from:
% http://www.ctan.org/pkg/ifpdf
% Also, note that IEEEtran.cls V1.7 and later provides a builtin
% \ifCLASSINFOpdf conditional that works the same way.
% When switching from latex to pdflatex and vice-versa, the compiler may
% have to be run twice to clear warning/error messages.


%%%%%%%%%%%%%%%%%%%%%%%%%%%%%%%%%%%%%%%%%%%%%%%%%%%%%%%%%%
% *** CITATION PACKAGES ***
%
\usepackage[nocompress]{cite}
% cite.sty was written by Donald Arseneau
% V1.6 and later of IEEEtran pre-defines the format of the cite.sty package
% \cite{} output to follow that of the IEEE. Loading the cite package will
% result in citation numbers being automatically sorted and properly
% "compressed/ranged". e.g., [1], [9], [2], [7], [5], [6] without using
% cite.sty will become [1], [2], [5]--[7], [9] using cite.sty. cite.sty's
% \cite will automatically add leading space, if needed. Use cite.sty's
% noadjust option (cite.sty V3.8 and later) if you want to turn this off
% such as if a citation ever needs to be enclosed in parenthesis.
% cite.sty is already installed on most LaTeX systems. Be sure and use
% version 5.0 (2009-03-20) and later if using hyperref.sty.
% The latest version can be obtained at:
% http://www.ctan.org/pkg/cite
% The documentation is contained in the cite.sty file itself.
%
% Note that some packages require special options to format as the Computer
% Society requires. In particular, Computer Society  papers do not use
% compressed citation ranges as is done in typical IEEE papers
% (e.g., [1]-[4]). Instead, they list every citation separately in order
% (e.g., [1], [2], [3], [4]). To get the latter we need to load the cite
% package with the nocompress option which is supported by cite.sty v4.0
% and later.


%%%%%%%%%%%%%%%%%%%%%%%%%%%%%%%%%%%%%%%%%%%%%%%%%%%%%%%%%%
% When using XML fragments, using pretty-print is helpful.
%
\usepackage{listings}
% \usepackage{color}
% \definecolor{gray}{rgb}{0.4,0.4,0.4}
% \definecolor{darkblue}{rgb}{0.0,0.0,0.6}
%\definecolor{maroon}{rgb}{0.5,0,0}
% \definecolor{cyan}{rgb}{0.0,0.6,0.6}

\lstset{
  basicstyle=\ttfamily,
  columns=fullflexible,
  showstringspaces=false,
  commentstyle=\color{gray}\upshape,
  escapeinside={||},
  mathescape=true
}

\lstdefinelanguage{XML}
{
basicstyle=\ttfamily\footnotesize,
  morestring=[b]",
  moredelim=[s][\bfseries\color{Maroon}]{<}{\ },
  moredelim=[s][\bfseries\color{Maroon}]{</}{>},
  moredelim=[l][\bfseries\color{Maroon}]{/>},
  moredelim=[l][\bfseries\color{Maroon}]{>},
  morecomment=[s]{<?}{?>},
  morecomment=[s]{<!--}{-->},
  commentstyle=\color{gray},
  stringstyle=\color{blue},
  identifierstyle=\color{red}
%  morekeywords={type,id,value,impl}% list your attributes here
}
%
%---------------------------------------------------------


%%%%%%%%%%%%%%%%%%%%%%%%%%%%%%%%%%%%%%%%%%%%%%%%%%%%%%%%%%
% Better control over verbatim text
\usepackage{moreverb}

%%%%%%%%%%%%%%%%%%%%%%%%%%%%%%%%%%%%%%%%%%%%%%%%%%%%%%%%%%
% for syntax/grammar
\usepackage[nounderscore]{syntax}

%%%%%%%%%%%%%%%%%%%%%%%%%%%%%%%%%%%%%%%%%%%%%%%%%%%%%%%%%%
\usepackage[pdftex]{graphicx}
% declare the path(s) where your graphic files are
\graphicspath{{./figures/}}
% and their extensions so you won't have to specify these with
% every instance of \includegraphics
\DeclareGraphicsExtensions{.pdf}
% graphicx was written by David Carlisle and Sebastian Rahtz. It is
% required if you want graphics, photos, etc. graphicx.sty is already
% installed on most LaTeX systems. The latest version and documentation
% can be obtained at: 
% http://www.ctan.org/pkg/graphicx
% Another good source of documentation is "Using Imported Graphics in
% LaTeX2e" by Keith Reckdahl which can be found at:
% http://www.ctan.org/pkg/epslatex
%
% latex, and pdflatex in dvi mode, support graphics in encapsulated
% postscript (.eps) format. pdflatex in pdf mode supports graphics
% in .pdf, .jpeg, .png and .mps (metapost) formats. Users should ensure
% that all non-photo figures use a vector format (.eps, .pdf, .mps) and
% not a bitmapped formats (.jpeg, .png). The IEEE frowns on bitmapped formats
% which can result in "jaggedy"/blurry rendering of lines and letters as
% well as large increases in file sizes.
%
% You can find documentation about the pdfTeX application at:
% http://www.tug.org/applications/pdftex

% Definitions for placeholder figures
\newcommand{\dummyfigX}{\fbox{\parbox[h][1.75in][t]{0.95\textwidth}{\emph{Placeholder}}}} % full page width (figure*)
\newcommand{\dummyfigXX}{\fbox{\parbox[h][1.75in][t]{0.95\columnwidth}{\emph{Placeholder}}}} % 1 column width (in 2 column format)
\newcommand{\dummyfigXXX}{\fbox{\parbox[h][1.75in][t]{0.30\textwidth}{\emph{Placeholder}}}} % 1/3 full page width (figure*)
\newcommand{\dummyfigXXXX}{\fbox{\parbox[h][1.75in][t]{0.23\textwidth}{\emph{Placeholder}}}} % 1/4 full page width (figure*)


%%%%%%%%%%%%%%%%%%%%%%%%%%%%%%%%%%%%%%%%%%%%%%%%%%%%
% *** MATH PACKAGES ***
%
\usepackage[cmex10]{amsmath}
\usepackage{amssymb}
\usepackage{mathtools}
\usepackage{amsthm}
\usepackage{amsfonts}
\newtheorem{cons}{Constraint}
%
% A popular package from the American Mathematical Society that provides
% many useful and powerful commands for dealing with mathematics.
%
% Note that the amsmath package sets \interdisplaylinepenalty to 10000
% thus preventing page breaks from occurring within multiline equations. Use:
%\interdisplaylinepenalty=2500
% after loading amsmath to restore such page breaks as IEEEtran.cls normally
% does. amsmath.sty is already installed on most LaTeX systems. The latest
% version and documentation can be obtained at:
% http://www.ctan.org/pkg/amsmath



%%%%%%%%%%%%%%%%%%%%%%%%%%%%%%%%%%%%%%%%%%%%%%%%%%%%
% *** SUBFIGURE PACKAGES ***
\usepackage{subfig} %[caption=false,font=footnotesize,labelfont=sf,textfont=sf]
%
% subfig.sty, written by Steven Douglas Cochran, is the modern replacement
% for subfigure.sty, the latter of which is no longer maintained and is
% incompatible with some LaTeX packages including fixltx2e. However,
% subfig.sty requires and automatically loads Axel Sommerfeldt's caption.sty
% which will override IEEEtran.cls' handling of captions and this will result
% in non-IEEE style figure/table captions. To prevent this problem, be sure
% and invoke subfig.sty's "caption=false" package option (available since
% subfig.sty version 1.3, 2005/06/28) as this is will preserve IEEEtran.cls
% handling of captions.
% Note that the Computer Society format requires a sans serif font rather
% than the serif font used in traditional IEEE formatting and thus the need
% to invoke different subfig.sty package options depending on whether
% compsoc mode has been enabled.
%
% The latest version and documentation of subfig.sty can be obtained at:
% http://www.ctan.org/pkg/subfig




% *** SPECIALIZED LIST PACKAGES ***
%
\usepackage{algorithmicx}
\usepackage{algpseudocode}
\usepackage[ruled]{algorithm}
\definecolor{light-gray}{gray}{0.75}
\algrenewcommand{\algorithmiccomment}[1]{\hskip3em{{\footnotesize \textcolor{light-gray}{$\blacktriangleright$}}} #1}
%
% This package provides an algorithmic environment fo describing algorithms.
% You can use the algorithmic environment in-text or within a figure
% environment to provide for a floating algorithm. 


%%%%%%%%%%%%%%%%%%%%%%%%%%%%%%%%%%%%%%%%%%%%%%%%%%%%%%%%%%
% Table relates packages
\usepackage{multirow} % Multi-row tables
\usepackage{rotating} % sideways table
\usepackage{booktabs} % better lines
\usepackage{colortbl} % cell colors
\usepackage{tablefootnote} % add support for footnote in table

%%%%%%%%%%%%%%%%%%%%%%%%%%%%%%%%%%%%%%%%%%%%%%%%%%%%%%%%%%
% *** PDF, URL AND HYPERLINK PACKAGES ***
%
%\usepackage[colorlinks,bookmarksopen,bookmarksnumbered,citecolor=red,urlcolor=red]{hyperref}
\usepackage[pdftex,colorlinks=true,urlcolor=blue,citecolor=blue]{hyperref}


%%%%%%%%%%%%%%%%%%%%%%%%%%%%%%%%%%%%%%%%%%%%%%%%%%%%%%%%%%
% Avoids contiguous empty spaces
\usepackage{xspace}

%%%%%%%%%%%%%%%%%%%%%%%%%%%%%%%%%%%%%%%%%%%%%%%%%%%%%%%%%%
% IEEETrans class fix for enumitem. provide for legacy IED commands/lengths when possible
% http://comments.gmane.org/gmane.editors.lyx.general/68611
\let\labelindent\relax
\usepackage{enumitem}


%%%%%%%%%%%%%%%%%%%%%%%%%%%%%%%%%%%%%%%%%%%%%%%%%%%%%%%%%%
% correct bad hyphenation here
\hyphenation{compu-ta-tio-nal}

% define repetitive complex fragments here
\newcommand{\floe}{$\mathcal{F}{\ell}{o}{\varepsilon}$\xspace}
\newcommand{\prc}{\mathcal{P}}
\newcommand{\chn}{\mathcal{C}}


%%%%%%%%%%%%%%%%%%%%%%%%%%%%%%%%%%%%%%%%%%%%%%%%%%%%%%%%%%
% generate lorum ipsum placeholder text
%\usepackage[english]{babel}
\usepackage{blindtext}
\newcommand{\blindtextc}{\color{gray}\blindtext\color{black}}
\newcommand{\Blindtextc}{\color{gray}\Blindtext\color{black}}




\begin{document}

\title{Distributed Knowledge Graph Querying on \\ Edge and Cloud}


\author{\IEEEauthorblockN{Shriram R.}\\
\IEEEauthorblockA{Department of Computational and Data Sciences\\
Indian Institute of Science, Bangalore 560012 INDIA\\
Email: shriramr@iisc.ac.in}
}


\maketitle


\begin{abstract}
In this mid-term period, a basic distributed knowledge graph querying system has been implemented by adopting an existing graph database engine with functionalities to fetch and store local knowledge graph, partition queries for local and remote graph database, combining results from local and remote for vertex search, edge search and shortest path search queries. Experiments were performed using a small dataset to study the performance of different types of queries and an analysis of result has been performed.  
\end{abstract}

\section{System Design and Implementation}

The following sections cover the system design and implementation completed so far with respect to the proposed targets for midterm,

\subsection{Remote (Cloud) Layer}
This layer consists of an in-memory graph database engine \emph{TinkerGraph}\cite{tinkergraph} running on the head node of \emph{Rigel} Cluster. Integration of \emph{GoDB} \cite{jamadagni:ccgrid:2016} was explored but due to technical issues, \emph{TinkerGraph} was chosen as the desired engine. It offers a good set of APIs to run a variety of queries related to graphs and supports different programming languages. It is built on top of \emph{TinkerPop} framework and \emph{Gremlin}\cite{Rodriguez:2015:GGT:2815072.2815073} programming language A distributed version spanning multiple nodes will be explored in future.

\subsection{Edge Layer}
The edge layer consists of different modules each with a specific functionality. The modules are explained in detail in the section below,

\subsubsection{Edge Graph Processing}
The same \emph{TinkerGraph} engine used in the remote layer is spawned on the edge as a single node in-memory store. The performance of this is evaluated through experiments. A custom engine is required only if the edge based \emph{TinkerGraph} engine is found to be a bottleneck.

\subsubsection{Knowledge Graph Partitioning}
The logic for partitioning is based on the association of an edge with a single entity in the Knowledge graph. E.g. <India>. Given a entity, this module queries the remote server to fetch the subgraph centered at the given entity and spanning for a specified number of hops (E.g. 2). This subgraph is then inserted into the \emph{TinkerGraph} server running locally and forms the local knowledge graph. \ref{fig:lkg} shows a snapshot example of local knowledge graph centered around <India>.

\begin{figure}[!t]
	\centering
	\includegraphics[width=0.95\columnwidth]{India.png}
	\caption{Local Knowledge Graph}
	\label{fig:lkg}
\end{figure}

\subsubsection{Query Partitioning}
The query partitioning module divides the input query into local and remote queries and fires them against the respective servers. The input queries are provided in JSON format with custom fields for each query type. The logic is different for each type of query. It is detailed below,
\begin{itemize}%[noitemsep,topsep=0pt,parsep=0pt,partopsep=0pt]
	\item \emph{Vertex Search} - The given query consists of three fields one for label based search, one for set of vertices with incoming edge from a given vertex and one for set of vertices with outgoing edge to a given vertex. The query is unmodified and used for local and remote since this search operation is embarrassingly parallel
	\item \emph{Edge Search} - The given query is unmodified as in the previous case since it also embarrassingly parallel and consists of three fields one for label based search, one for all edges outgoing from a vertex and other for all edges incoming to a vertex
	\item \emph{Path Search} - The path search queries consists of a source vertex and target vertex and finds the shortest path between them in an undirected knowledge graph. It can have three result types: Full path is in edge layer, full path is in remote layer and path crosses edge and remote layer. The logic follows these steps,
	\begin{itemize}
		\item A local query is fired first to search for path completely contained in the edge layer
		\item If the previous step returns no path, then a series of queries are fired to the remote layer with source vertex changed to \emph{cut vertices} and target vertex unchanged. This will return a portion of shortest path in the remote server
		\item A series of local queries are fired to determine the paths from original source to \emph{cut vertices}. 
	\end{itemize}
Some example queries of each type is given below,
\begin{verbatim}
{ // Vertex Search
    "type": "vertex_search",
    "filter": {
        "has_label": "<50_Cent>",
        "has": null,
        "from": null,
        "to": null
    }
}
{ // Edge Search
    "type": "edge_search",
    "filter": {
        "has_label": "<exports>",
        "from": null,
        "to": null
    }
}
{ // Path Search
    "type": "path_search",
    "filter": {
        "from": "<India>",
        "to": "<Barack_Obama>"
    }
}
\end{verbatim}
\end{itemize}

\subsubsection{Combining Query Results}
Combining the query results is implemented as follows,
\begin{itemize}
	\item \emph{Vertex \& Edge Search} - The result set from local and remote server search are combined using \emph{set union} operation
	\item \emph{Path search} - If the path is entirely contained inside edge, the local server result is directly used. For cross paths, The result from local server and remote server are joined at the \emph{cut vertex} and then the shortest path is determined and provided as output
\end{itemize}

It has to be noted that all custom modules in edge layer were built in \emph{Python} and queries were submitted to local and remote servers through \emph{gremlinpython} package. No indexes were used in this mid term project. Also, the queries to remote server can performed concurrently due to an issue with \emph{gremlin}. So, only one local query and one remote query can run concurrently at a time.

%\begin{figure*}[th]
%	\centering%~%
%	\subfloat[High Level Architecture]{
%		\includegraphics[height=0.23\textheight]{5.png}
%		\label{fig:arch}
%	}\qquad
%	\subfloat[Logical Flow]{
%		\includegraphics[height=0.15\textheight]{6.png}
%		\label{fig:logic-flow}
%	}\\
%	\label{fig:problem-approach}
%	\caption{Proposed Design}
%	\vspace{-0.1in}
%\end{figure*} 


\section{Experiments}

The experimental setup is as follows: A single node in-memory version of \emph{TinkerGraph} was run in the head node of \emph{Rigel} cluster. The node has the following specs: 32 core AMD Opteron 6376 processor with 128 GB DDR3 RAM running CentOS 7 version. It is connected through 1 GBps ethernet and has 6 TB HDD and 256 GB SSD storage. Persistence was turned off in this layer.

A single edge device was spawned as Docker container with 4 cores, 1GB RAM and Ubuntu 18.04 LTS simulating a Raspberry Pi3B+ class of device. The network connectivity has a latency of 5ms and bandwidth of 100 MBps.

A smaller version of the \emph{YAGO}\cite{Suchanek:2007:YCS:1242572.1242667} dataset was used in these experiments. It consists of 18845 edges, 14977 vertices and 1400 total no. of vertex attributes. The vertices consists of most popular entities in the knowledge graph.

There were some technical issues in converting the large dataset into a compatible version for \emph{TinkerGraph}. This will be explored as part of final term project. The local knowledge graph was centered at <India> with 2 hop neighbourhood.

\subsection{Vertex Search}

Random queries with different values for the search fields uniformly distributed were generated for 500s and the time taken for local and remote query response was measured. \ref{fig:vertex_search} shows the time taken distribution for each query type with X-axis denoting the query type and Y-axis denoting the time taken in ms (log). It can be observed that the time taken by local queries are generally less compared to remote queries (approx. $5$x). This is as expected since there is a network latency and size of remote knowledge graph slowing the remote server query time. The time taken is consistent for all query types under a given server. This is as expected since the query requires maximum of 1 hop distance traversal. 

\begin{figure}[!t]
	\centering
	\includegraphics[width=0.95\columnwidth]{vs.pdf}
	\caption{Vertex Search - Performance}
	\label{fig:vertex_search}
\end{figure} 

\subsection{Edge Search}

Similar to Vertex search, random queries were generated with different values in search fields uniformly for 500s. \ref{fig:edge_search} shows the time taken in ms for different query types. The remote server query times are order of magnitudes higher than that of local which is as expected. The label search takes longest time since there are many edges having the same label unlike vertex where labels are unique. From and To searches will hit only a single vertex and its one hop neighbour and so their performance is similar to that of vertex search

\begin{figure}[!t]
	\centering
	\includegraphics[width=0.95\columnwidth]{es.pdf}
	\caption{Edge Search - Performance}
	\label{fig:edge_search}
\end{figure} 

\subsection{Path Search}

For Path search, the source vertex was fixed at <India> and the destination vertex was randomly assigned from the list of all vertices. The experiment was run for 1000s to allow for more data points. The no. of \emph{cut vertices} was limited to 10 so that experiments complete in reasonable time. This means there could be some incorrect results. \ref{fig:path_search} shows the time taken in ms for different query types. Remote server took much larger time to respond since a series of queries were fired (one for each \emph{cut vertex}). The time taken also depended on the length of paths returned in the result.


\begin{figure}[!t]
	\centering
	\includegraphics[width=0.95\columnwidth]{ps.pdf}
	\caption{Path Search - Performance}
	\label{fig:path_search}
\end{figure}

\section{Conclusion}
Regarding the mid-term deliverables, key modules to make edge layer work were implemented successfully. However, these modules have to be extended and refined to handle different other query types and possible corner cases. One of the major item is missed in this deliverable is the non-availability of experiments using the full dataset which could clearly explain the scalability of the system.


% trigger a \newpage just before the given reference
% number - used to balance the columns on the last page
% adjust value as needed - may need to be readjusted if
% the document is modified later
%\IEEEtriggeratref{6}
% The "triggered" command can be changed if desired:
%\IEEEtriggercmd{\enlargethispage{-5in}}

% references section

% can use a bibliography generated by BibTeX as a .bbl file
% BibTeX documentation can be easily obtained at:
% http://mirror.ctan.org/biblio/bibtex/contrib/doc/
% The IEEEtran BibTeX style support page is at:
% http://www.michaelshell.org/tex/ieeetran/bibtex/
\bibliographystyle{IEEEtran}
% argument is your BibTeX string definitions and bibliography database(s)
\bibliography{paper}

% that's all folks
\end{document}

%%% Local Variables:
%%% mode: latex
%%% TeX-master: t
%%% End:
